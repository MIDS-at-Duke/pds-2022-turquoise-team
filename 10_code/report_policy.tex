% Options for packages loaded elsewhere
\PassOptionsToPackage{unicode}{hyperref}
\PassOptionsToPackage{hyphens}{url}
%
\documentclass{article}
\usepackage{hyperref}
\usepackage{fancyhdr}
\usepackage{amssymb,amsmath}
\usepackage{ifxetex,ifluatex}
\ifnum 0\ifxetex 1\fi\ifluatex 1\fi=0 
  \usepackage[T1]{fontenc}
  \usepackage[utf8]{inputenc}
  \usepackage{textcomp}
\else
  \usepackage{unicode-math}
  \defaultfontfeatures{Scale=MatchLowercase}
  \defaultfontfeatures[\rmfamily]{Ligatures=TeX,Scale=1}
\fi
% Use upquote if available, for straight quotes in verbatim environments
\IfFileExists{upquote.sty}{\usepackage{upquote}}{}
\IfFileExists{microtype.sty}{% use microtype if available
  \usepackage[]{microtype}
  \UseMicrotypeSet[protrusion]{basicmath} % disable protrusion for tt fonts
}{}
\makeatletter
\@ifundefined{KOMAClassName}{% if non-KOMA class
  \IfFileExists{parskip.sty}{%
    \usepackage{parskip}
  }{% else
    \setlength{\parindent}{0pt}
    \setlength{\parskip}{6pt plus 2pt minus 1pt}}
}{% if KOMA class
  \KOMAoptions{parskip=half}}
\makeatother
\usepackage{xcolor}
\IfFileExists{xurl.sty}{\usepackage{xurl}}{} % add URL line breaks if available
\IfFileExists{bookmark.sty}{\usepackage{bookmark}}{\usepackage{hyperref}}
\hypersetup{
  hidelinks,
  pdfcreator={LaTeX via pandoc}}
\urlstyle{same} % disable monospaced font for URLs
\usepackage{longtable,booktabs}
% Correct order of tables after \paragraph or \subparagraph
\usepackage{etoolbox}
\makeatletter
\patchcmd\longtable{\par}{\if@noskipsec\mbox{}\fi\par}{}{}
\makeatother
% Allow footnotes in longtable head/foot
\IfFileExists{footnotehyper.sty}{\usepackage{footnotehyper}}{\usepackage{footnote}}
\makesavenoteenv{longtable}
\usepackage{graphicx}
\makeatletter
\def\maxwidth{\ifdim\Gin@nat@width>\linewidth\linewidth\else\Gin@nat@width\fi}
\def\maxheight{\ifdim\Gin@nat@height>\textheight\textheight\else\Gin@nat@height\fi}
\makeatother
% Scale images if necessary, so that they will not overflow the page
% margins by default, and it is still possible to overwrite the defaults
% using explicit options in \includegraphics[width, height, ...]{}
\setkeys{Gin}{width=\maxwidth,height=\maxheight,keepaspectratio}
% Set default figure placement to htbp
\makeatletter
\def\fps@figure{htbp}
\makeatother
\setlength{\emergencystretch}{3em} % prevent overfull lines
\providecommand{\tightlist}{%
  \setlength{\itemsep}{0pt}\setlength{\parskip}{0pt}}
\setcounter{secnumdepth}{-\maxdimen} % remove section numbering


%%%% PROJECT TITLE
\title{Estimate the Impact of Opioid Control Policies (for policymakers)
\\
}

\author{\href{mailto:beibei.du@duke.edu}{Beibei Du}, 
\href{mailto:wafiakmal.miftah@duke.edu}{Wafiakmal Miftah},
\href{mailto:suzanna.thompson@duke.edu}{Suzanna Thompson},
\href{mailto:alisa.tian@duke.edu}{Alisa Tian}
}
\date{December 12, 2022}


\begin{document}
\maketitle
% \thispagestyle{firstpage}
% \begin{document}
\section{Motivation}

For this project, we are trying to address the problem of rising opioid
usage and its connection with death by drugs in the United States. We
have two specific questions: What effect did the opioid policies in
Texas, Washington, and Florida have on the number of opioid deaths and
overall drug deaths in those states? In other words, how do opioid
deaths and overall drug deaths differ in Texas, Washington, and Florida
from our chosen control states? We will mention how we decide the
control states in the OverView of the Data part.

To change if the policy intervention is effective is vital for the
policymakers to improve on the past enactment of the policies on opioid
usage. There are many papers concerning the drug overdose problem since
opioid is prescribed as a medication for chronic pain. By the
neurological property of opioids, the neurotransmitters acetylcholine
and dopamine are released, which controls people's Central Nervous
System (CNS) and Peripheral Nervous System(PNS) and makes people feel
happy accordingly (Janice C. Froehlich). On the other hand, an overdose
can lead to a lot of severe problems, including death. Thus, to minimize
the possibility of overdose, we try to answer the two questions
mentioned above.

\section{Overview of the Data}

Upon choosing the control states, we aim to select the states with
similar social-demographic characteristics. The following factors are
considered: state population, education ranking, average income, and
drug-related death rate, as well as the state's policy response to the
opioid crisis. We choose the individual treatment state as the baseline.
For example, when calculating the state's average income based on that
of Florida, we divide the individual state's population by Florida's
population. A percentage close to 100\% means the state's population is
similar to Florida's.

After calculating the similarity for these five indicators (state
population, education ranking, average income, rate of drug-related
death, and state's policy response to opioid crisis), the States are
chosen by a combined weighted metric. Percentages are summed and divided
by 4. The population is over-impacting the combined metric for Texas, so
it is excluded. Our control states are decided based on if they are in
the interval (80\%, 120\%). Also, due to the county designations
confusion in Alaska in 2010, we have excluded Alaska from our control
states.

The list of our treatment states and accompanying control states:

\begin{longtable}[]{@{}ll@{}}
\toprule
Treatment State & Control States\tabularnewline
\midrule
\endhead
Texas & Arkansas, California, Georgia, Missouri, New York,
Wyoming\tabularnewline
Washington & Hawaii, Iowa, Kansas, Maine, Massachusetts, Minnesota,
Montana, Nebraska, North Dakota, Oregon, South Dakota, Virginia, and
Wyoming\tabularnewline
Florida & California, Nevada, New York\tabularnewline
\bottomrule
\end{longtable}

The source of our data are:

\begin{enumerate}
\def\labelenumi{\arabic{enumi}.}
\item
  \begin{quote}
  Opioids Prescription/Shipment data (Washington Post):
  \end{quote}
\end{enumerate}

\begin{quote}
In this dataset, it holds all the information of the opioids
prescription across the years in all states in the United States. This
dataset is about 100Gb and we select the three states and their
controlled states to preprocess. The unit of observation for this
dataset is at the county level. We sum all the data of all counties in a
state and take that into our analyses.
\end{quote}

\begin{enumerate}
\def\labelenumi{\arabic{enumi}.}
\setcounter{enumi}{1}
\item
  \begin{quote}
  Mortality Rate data (CDC Wonder):
  \end{quote}
\end{enumerate}

\begin{quote}
In this dataset, we have state, county, year, opioid-related cause of
death and the number of deaths per county and state. The unit of
observation for this data is at the county level.
\end{quote}

\begin{enumerate}
\def\labelenumi{\arabic{enumi}.}
\setcounter{enumi}{2}
\item
  \begin{quote}
  Population data (CDC Wonder):
  \end{quote}
\end{enumerate}

\begin{quote}
CDC is a reliable system for holding public health data and information
across the United States.
\end{quote}

For the shipment \& prescription question, we are excluding Texas as we
do not have an adequate sample of data.

\section{Analysis}

Our methodology for this project is pre-post and
difference-in-difference analysis.

\textbf{Hypothesis}:

We hypothesize that there is a causal effect between policy change in
opioid use and the case of overdose deaths in Florida, Washington, and
Texas State. We also hypothesize that while opioid prescriptions
decrease, drug-related deaths as a whole increase or remain stable. We
suggest this result as there is evidence to suggest that removing a drug
from an addict does not necessarily imply that the addict will recover.
In some cases, the addict will switch to a new drug of choice. In other
cases, the opioid addict will be in recovery and later overdose once
they're able to find an opioid. This is a well known phenomena as
addicts may forget that their tolerance for the opioid decreases during
times of recovery and then try to use the same opioid dose as they were
using active addiction. In turn, the addict will accidentally overdose.
We also suppose that opioid prescriptions decrease in Texas, Washington,
and Florida.

\hypertarget{interpretation-of-the-analysis-strengths-and-limitations}{%
\subsection{Interpretation of the Analysis (Strengths and
Limitations)}\label{interpretation-of-the-analysis-strengths-and-limitations}}

\hypertarget{florida}{%
\subsubsection{Florida}\label{florida}}

{Effects of Regulations on Opioid Shipments}

\includegraphics[width=2.87617in,height=2.19792in]{media/image1.png}\includegraphics[width=3.47292in,height=2.1875in]{media/image2.png}

Figure 1. \emph{Pre-Post and Diff-in-Diff Model for Opioid Shipments in
Florida and All Control States}

When we looked at the opioid prescription rate, it is obvious that
Figure 1 shows an upward trend before 2010, when the policy was
implemented. After its implementation in the year 2010, however, the
prescription per capita dropped significantly. We also compare it with
the control states: California, Nevada, and New York. We observe that
the prescription rate continues to increase after 2010, which is the
same as our assumption. This difference-in-difference analysis suggests
that the opioid policy in Florida is successful.

\includegraphics[width=3.23125in,height=2.54192in]{media/image3.png}\includegraphics[width=3.025in,height=2.4943in]{media/image4.png}

Figure 2. \emph{Pre-Post and Diff-in-Diff Model for Mortality Rate in
Florida and All Control States}

Figure 2 shows the pre-post death rate related to drug use per 100,000
people in Florida and the difference-in-difference graph for control
states (California, Nevada, and New York). Florida showed a downward
trend in the death rate before 2010, when the opioid policy was
implemented. The pre-post graph showed that the death rate tren line
dropped even steeper after implementing the policy. Meanwhile, the
difference-in-difference analysis showed that the death rate in the
control states decreased slightly. This decreasing graph for Florida
hints that the opioid policy had a positive impact on decreasing
drug-related death in Florida.

\hypertarget{texas}{%
\subsubsection{Texas}\label{texas}}

\includegraphics[width=3.17208in,height=2.46717in]{media/image5.png}\includegraphics[width=2.94375in,height=2.46895in]{media/image6.png}

Figure 3. \emph{Pre-Post and Diff-in-Diff Model for Mortality Rate in
Texas and All Control States}

Figure 3 shows the pre-post graph for the death rate related to drug use
per 100,000 people in Texas and the difference-in-difference graph for
control states (Arkansas, California, Georgia, Missouri, New York, and
Wyoming). Texas showed a downward trend in the death rate before 2007,
when the opioid policy was implemented. The pre-post graph showed that
after implementing the policy, the death rate remained upward and even
steeper from before. Meanwhile, the difference-in-difference analysis
showed that the death rate in the control states stagnated after policy
implementation in 2007. This increasing graph for Texas hints that the
opioid policy had a negative impact in decreasing drug-related death in
Texas.

\hypertarget{section}{%
\subsubsection{}\label{section}}

\hypertarget{washington}{%
\subsubsection{Washington}\label{washington}}

\includegraphics[width=3.12678in,height=2.25208in]{media/image7.png}\includegraphics[width=2.94583in,height=2.27584in]{media/image8.png}

Figure 4. \emph{Pre-Post and Diff-in-Diff Model for Mortality Rate in
Washington and All Control States}

Figure 4 shows the pre-post graph for the death rate related to drug use
per 100,000 people in Washington and the difference-in-difference graph
for control states (Hawaii, Iowa, Kansas, Maine, Massachusetts,
Minnesota, Montana, Nebraska, North Dakota, Oregon, South Dakota,
Virginia, Wyoming). Washington showed a stagnant trend in the death rate
before 2012, when the opioid policy was implemented. The pre-post graph
showed that after implementing the policy, the death rate decreased and
showed a downward trend, but not as steep as before. Meanwhile, the
difference-in-difference analysis showed that the death rate in the
control states was stagnant after policy implementation in 2012. This
slightly decreasing graph for Washington hints that the opioid policy
had a positive impact in decreasing drug-related death in Washington.

\includegraphics[width=2.80625in,height=2.20036in]{media/image9.png}\includegraphics[width=3.36979in,height=2.21019in]{media/image10.png}

Figure 5. \emph{Pre-Post and Diff-in-Diff Model for Opioid Shipments in
Washington and All Control States}

Figure 5 shows the pre-post graph for opioid shipment in Morphine Gram
Equivalent in Washington and the difference-in-difference graph for
control states (Hawaii, Iowa, Kansas, Maine, Massachusetts, Minnesota,
Montana, Nebraska, North Dakota, Oregon, South Dakota, Virginia,
Wyoming). Washington showed an upward trend of opioid shipment per
100,000 people before 2012, when the opioid policy was implemented. The
pre-post graph also showed the shipment rate in a stagnant trend after
implementing the policy. Meanwhile, the difference-in-difference
analysis showed that the opioid shipment in the control states also
stagnated after policy implementation in 2012. This stagnated graph for
Washington hints that the opioid policy had a slightly positive impact
on decreasing opioid shipments in Washington.

\section{Conclusion}

It is with great importance that you, the policymaker, should take
appropriate action when handling the opioid crisis. The greatest
positive impact from our analysis comes from Florida. Florida's death
rate and prescription \& shipment rate had the biggest change. The rate
of shipments \& prescriptions decreased as well as the rate of death. In
conclusion, if you, the policymaker, is seeking a state to model your
policy after, Florida is the best state to consider when serving your
constituents.

\section{References}

\begin{quote}
Centers for Disease Control and Prevention. (2022, September 6).
\emph{FASTSTATS - deaths and mortality}. Deaths and Mortality. Retrieved
November 10, 2022, from
\href{https://www.cdc.gov/nchs/fastats/deaths.htm}{{https://www.cdc.gov/nchs/fastats/deaths.htm}}

Home. Funding for Medication-Assisted Treatment (MAT) - Rural Community
Toolbox. (n.d.). Retrieved November 29, 2022, from
\href{https://www.ruralcommunitytoolbox.org/funding/topic/medication-assisted-treatment}{{https://www.ruralcommunitytoolbox.org/funding/topic/medication-assisted-treatment}}

Home Page. LHSFNA. (2022, October 3). Retrieved November 29, 2022, from
\href{https://www.lhsfna.org/}{{https://www.lhsfna.org/}}
\end{quote}

Mital, S., Wisdom, A. C., \& Wolff, J. G. (2022). Improving partnerships
between public health and public safety to reduce overdose deaths: An
inventory from the CDC overdose data to action funding initiative.
\emph{Journal of Public Health Management and Practice},
\emph{28}(Supplement 6).
\href{https://doi.org/10.1097/phh.0000000000001637}{{https://doi.org/10.1097/phh.0000000000001637}}

\begin{quote}
Opioid peptides - National Institutes of Health. (n.d.). Retrieved
November 29, 2022, from
\href{https://pubs.niaaa.nih.gov/publications/arh21-2/132.pdf}{{https://pubs.niaaa.nih.gov/publications/arh21-2/132.pdf}}

Wickramatilake, S., Zur, J., Mulvaney-Day, N., Klimo, M. C. von, Selmi,
E., \&amp; Harwood, H. (2017). How states are tackling the opioid
crisis. Public health reports (Washington, D.C. : 1974). Retrieved
November 29, 2022, from
\href{https://www.ncbi.nlm.nih.gov/pmc/articles/PMC5349480/}{{https://www.ncbi.nlm.nih.gov/pmc/articles/PMC5349480}}
\end{quote}

\end{document}
